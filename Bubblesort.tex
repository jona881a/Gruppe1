\documentclass[12pt,a4paper]{article}
\usepackage[danish]{babel}
\usepackage[utf8]{inputenc}
\usepackage{amsmath}
\usepackage{graphicx}

\begin{document}
\title{Boblesortering}
\author{
   Lars Atex, s010101
   \and
   Søren Tudent, s020202}
\date{1. oktober 2018}
\maketitle

\begin{abstract}
Dette dokument omhandler boblesortering. Der beskrives algoritmen og præsenteres en kompleksitetsanalyse.
\end{abstract}

\section{Introduktion}
Boblesortering (eng. bubble sort) er en populær sorteringsalgoritme og er en af de simpleste algoritmer at forstå og implementere. Dog er den ikke en særlig effektiv sorteringsalgoritme1; hverken for store eller små lister, og den anvendes sjældent i praksis. Boblesortering sorterer, som navnet antyder, elementerne i en liste ved at boble hvert element gennem listen til sin rette plads i listen.

\subsection{Pseudokode}
Wikipedia \cite{2} giver følgende pseudokode for boblesortering.

\begin{verbatim}
procedure bubbleSort( A : list of sortable items ) defined as:
  do
    swapped := false
    for each i in 0 to length(A) - 2 inclusive do:
      if A[i] > A[i+1] then
        swap( A[i], A[i+1] )
        swapped := true
      end if end for
    while swapped
end procedure
\end{verbatim}
En illustration af en kørsel af boblesortering fra Wikipedia kan ses på figur 1.





\begin{thebibliography}{2}

\bibitem{1}
 Donald Knuth,
 \textit{The Art of Computer Programming, Volume 3.}
 Addison- Wesley.

\bibitem{2}
 \textit{http://en.wikipedia.org/wiki/Bubble_sort}

\end{thebibliography}


\end{document}
